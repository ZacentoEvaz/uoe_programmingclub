\documentclass{article}

\usepackage{amsmath}
\usepackage[margin=1cm]{geometry}
\usepackage{graphicx}
\usepackage{hyperref}
\usepackage{listings}

\title{
    \textbf{Programming Club}\\
    Performing well at Programming Interviews
}
\author{Hugh Leather and Volker Seeker}
\date{10$^{th}$ October 2017}

\begin{document}
    \maketitle
   
    \section{How to handle a Question}

    \begin{enumerate}
        \item Pick your favourite programming language.
        \item Ask for requirements.
        \item Draft a solution (using pictures if possible).
        \item Create the function prototype.
        \item Do the programming.
        \item Test, test, test!
        \item Add error checking.
        \item Do time and space analysis.
    \end{enumerate}

    \section{General Hints}

    \begin{itemize}
        \item Be sure you know your favourite language in and out.
        \item Use short names for variables (just for the sake of writing faster).
        \item Do not turn around saying you are done before you actually did all the steps!
        \item Do not present a solution in an asking way by looking at the
            interviewer and searching for hints in his face.
        \item Do not assume requirements. Ask for them!
        \item Rather than asking for requirements, present possible options including
            pros and cons (show how smart you are).
        \item Make the actual unit test (table presentation).
    \end{itemize}

\end{document}
